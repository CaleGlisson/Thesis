\chapter{Introduction}
Computer systems are becoming more heterogenous in their nature, including not just a typical microprocessors, but also hardware accelerators in some form or another. This allows system designers to pick and choose which portions of their program will be accelerated by the hardware, while keeping non-critical components simple. 
One option in these systems is to use a Field Programmable Gate Array (FPGA) for the hardware portion, which allows for a Hardware Description Language (HDL) to be used to implement the hardware, rather than using discrete hardware accelerators. This means that in addition to the typical C or C++ code for the microprocessor, HDL code must be written for the FPGA acceleration as well as an interface between these two portions. While this gives the design incredible flexibility and control, the interface between the hardware and software is incredibly complex, and can create almost impossible to debug errors in any non trivial system. 